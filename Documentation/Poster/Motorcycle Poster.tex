%%%%%%%%%%%%%%%%%%%%%%%%%%%%%%%%%%%%%%%%%
% a0poster Portrait Poster 
% LaTeX Template
% with University Copenhagen logo
% Version 1.0 (22/06/13)
%
% Based on:
% The a0poster class was created by:
% Gerlinde Kettl and Matthias Weiser (tex@kettl.de)
% 
% This template has been downloaded from:
% http://www.LaTeXTemplates.com
%
%%%%%%%%%%%%%%%%%%%%%%%%%%%%%%%%%%%%%%%%%

%----------------------------------------------------------------------------------------
%	PACKAGES AND OTHER DOCUMENT CONFIGURATIONS
%----------------------------------------------------------------------------------------

\documentclass[a0,portrait]{a0poster}
\usepackage[utf8]{inputenc}

\usepackage{multicol} % This is so we can have multiple columns of text side-by-side
\columnsep=100pt % This is the amount of white space between the columns in the poster
\columnseprule=3pt % This is the thickness of the black line between the columns in the poster

\usepackage[svgnames]{xcolor} % Specify colors by their 'svgnames', for a full list of all colors available see here: http://www.latextemplates.com/svgnames-colors

\usepackage{times} % Use the times font
%\usepackage{palatino} % Uncomment to use the Palatino font
\usepackage{array}
\usepackage{graphicx} % Required for including images
\graphicspath{{figures/}} % Location of the graphics files
\usepackage{booktabs} % Top and bottom rules for table
\usepackage[font=small,labelfont=bf]{caption} % Required for specifying captions to tables and figures
\usepackage{amsfonts, amsmath, amsthm, amssymb} % For math fonts, symbols and environments
\usepackage{wrapfig} % Allows wrapping text around tables and figures

\usepackage{hyperref}
\hypersetup{colorlinks=true,
	citecolor=black,
	linkcolor=black, % links to parts of the document (e.g. index) in black
	urlcolor=blue} % links to resources outside the document in blue

\definecolor{ku}{RGB}{144,26,30}
\definecolor{ku-yellow}{RGB}{255,249,25}

 \usepackage{eso-pic}
               \newcommand\BackgroundIm{
               \put(66,-71){
               \parbox[b][\paperheight]{\paperwidth}{%
               \vfill
               \centering
               \includegraphics[height=\paperheight,width=\paperwidth,
               keepaspectratio]{UWTSD-house.pdf}%
               \vfill
               }}}

\begin{document}
\AddToShipoutPicture*{\BackgroundIm}
%----------------------------------------------------------------------------------------
%	POSTER HEADER 
%----------------------------------------------------------------------------------------

% The header is divided into two boxes:
% The first is 75% wide and houses the title, subtitle, names, university/organization and contact information
% The second is 25% wide and houses a logo for your university/organization or a photo of you
% The widths of these boxes can be easily edited to accommodate your content as you see fit



\begin{minipage}[t]{0.60\linewidth}
	\vspace{9.5cm}
	\Huge \color{ku} \textbf{A Risky Coexistence: Examining the Challenges Faced by Autonomous Vehicles and Motorcycles} \color{Black}\\ % Title
	\huge\textit{Research of the Enhancing Self-Driving Car Performance: The
		Potential Dangers of Autonomous Vehicles and Motorcycles}\\[1cm] % Subtitle
	\Large \textbf{Edward S. R. Patch (1801492)}\\[0.5cm] % Author(s)
	\Large Software Engineering and Artificial Intelligence\\[0.4cm] % University/organization

\end{minipage}
%
\begin{minipage}[t]{0.40\linewidth}
	\vspace{9.5cm}
	\flushright
	\color{DarkSlateGray}
	\Large \textbf{Contact Information:}\\
	Waterfront IQ Campus\\
	University of Wales Trinity,\\
	Swansea, SA1 8EW.\\[1cm]
	Email: \texttt{1801492@student.uwtsd.ac.uk} % Email address
\end{minipage}

\vspace{1cm} % A bit of extra whitespace between the header and poster content

%----------------------------------------------------------------------------------------

\begin{multicols}{2} % This is how many columns your poster will be broken into, a portrait poster is generally split into 2 columns

	%----------------------------------------------------------------------------------------
	%	ABSTRACT
	%----------------------------------------------------------------------------------------

	\color{ku} % Navy color for the abstract

	\begin{abstract}

	\end{abstract}

	%----------------------------------------------------------------------------------------
	%	INTRODUCTION
	%----------------------------------------------------------------------------------------

	\color{DarkRed} % SaddleBrown color for the introduction

	\section*{Introduction}

	
	%----------------------------------------------------------------------------------------
	%	OBJECTIVES
	%----------------------------------------------------------------------------------------

	\color{DarkSlateGray} % DarkSlateGray color for the rest of the content

	\section*{Main Objectives}
		\begin{enumerate}
			\item Understanding of what dangers exist with AVs.
			\item Establishing the appropriate datasets to train and test the models.
			\item Pre-processing any datasets to improve the training progress.
			\item Evaluation of the test results to see specific information about where AVs may fail.
		\end{enumerate}

	%----------------------------------------------------------------------------------------
	%	MATERIALS AND METHODS
	%----------------------------------------------------------------------------------------

	\section*{Materials and Methods}
		\subsection*{Materials and Methods | Training}
			With being faced with the challenge of setting up a high-end model equivalent to a leading AV manufacturer, similarly to Tesla, then it is very important to use quality Object Classification training material. Using the Qualititave Research method with the use of video frames and labels to classify the different objects found in the video is required. Using different sources found in various research journals, Roboflow and other materials are 
		\subsection*{Materials and Methods | Testing}
			Testing materials must include video content, split into multiple frames To test the trained YOLO model, with enough images to create a strong argument. Joining a motorcycle group and exploring various routes across the United Kingdom, including motorways, dual carriageways, A-roads, and backroads, with motorcycles overtaking, filtering, and navigating blindspots, can lead to unexplored scenarios and questions that may have been previously overlooked.
				
			A decided factor is to use a Drift Innovation Ghost XL motorcycle camera attached to a motorcycle that rides within the group, then swap the camera with another rider after some time. This way, when combining the content helps with identifying how Object Classification copes with numerous blindspots and draw some questions to further the research concerning the current safety of AV vehicles. 
			
			Two sports bikes and two cruisers are selected for material to test how Object Classification models handle different motorcycle styles. Ideal footage would include Scramblers, Trikes and other similar vehicles to establish how Object Classification models work in an estimated manner. A perfect material would be that during the ride out, conducted on 18$^\text{th}$ July 2023, Tuesday, would capture these vehicles, which either pass by or join us in sections of the rides. The group is instructed to overtake and be undertaken by the camera vehicle to create plenty of footage to put the YOLO model to the test. However, it is worth noting that no rider is pressured into doing anything illegal or unsafe.
	%----------------------------------------------------------------------------------------
	%	RESULTS 
	%----------------------------------------------------------------------------------------

	\section*{Results}


	%----------------------------------------------------------------------------------------
	%	CONCLUSIONS
	%----------------------------------------------------------------------------------------

	\color{DarkRed} % SaddleBrown color for the conclusions to make them stand out

	\section*{Conclusions}


	\color{DarkSlateGray} % Set the color back to DarkSlateGray for the rest of the content

	%----------------------------------------------------------------------------------------
	%	FORTHCOMING RESEARCH
	%----------------------------------------------------------------------------------------

	\section*{Forthcoming Research}

	%----------------------------------------------------------------------------------------
	%	REFERENCES
	%----------------------------------------------------------------------------------------

	\nocite{*} % Print all references regardless of whether they were cited in the poster or not
	\bibliographystyle{plain} % Plain referencing style
	\bibliography{ref} % Use the example bibliography file sample.bib

	%----------------------------------------------------------------------------------------
	%	ACKNOWLEDGEMENTS
	%----------------------------------------------------------------------------------------

	\section*{Acknowledgements}

	%----------------------------------------------------------------------------------------

\end{multicols}
\end{document}
