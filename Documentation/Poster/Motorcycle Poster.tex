%%%%%%%%%%%%%%%%%%%%%%%%%%%%%%%%%%%%%%%%%
% a0poster Portrait Poster 
% LaTeX Template
% with University Copenhagen logo
% Version 1.0 (22/06/13)
%
% Based on:
% The a0poster class was created by:
% Gerlinde Kettl and Matthias Weiser (tex@kettl.de)
% 
% This template has been downloaded from:
% http://www.LaTeXTemplates.com
%
%%%%%%%%%%%%%%%%%%%%%%%%%%%%%%%%%%%%%%%%%

%----------------------------------------------------------------------------------------
%	PACKAGES AND OTHER DOCUMENT CONFIGURATIONS
%----------------------------------------------------------------------------------------

\documentclass[a0,portrait]{a0poster}
\usepackage[utf8]{inputenc}

\usepackage{multicol} % This is so we can have multiple columns of text side-by-side
\columnsep=100pt % This is the amount of white space between the columns in the poster
\columnseprule=3pt % This is the thickness of the black line between the columns in the poster

\usepackage[svgnames]{xcolor} % Specify colors by their 'svgnames', for a full list of all colors available see here: http://www.latextemplates.com/svgnames-colors

\usepackage{times} % Use the times font
%\usepackage{palatino} % Uncomment to use the Palatino font
\usepackage{array}
\usepackage{graphicx} % Required for including images
\graphicspath{{figures/}} % Location of the graphics files
\usepackage{booktabs} % Top and bottom rules for table
\usepackage[font=small,labelfont=bf]{caption} % Required for specifying captions to tables and figures
\usepackage{amsfonts, amsmath, amsthm, amssymb} % For math fonts, symbols and environments
\usepackage{wrapfig} % Allows wrapping text around tables and figures

\usepackage{hyperref}
\hypersetup{colorlinks=true,
	citecolor=black,
	linkcolor=black, % links to parts of the document (e.g. index) in black
	urlcolor=blue} % links to resources outside the document in blue

\definecolor{ku}{RGB}{144,26,30}
\definecolor{ku-yellow}{RGB}{255,249,25}

 \usepackage{eso-pic}
               \newcommand\BackgroundIm{
               \put(66,-71){
               \parbox[b][\paperheight]{\paperwidth}{%
               \vfill
               \centering
               \includegraphics[height=\paperheight,width=\paperwidth,
               keepaspectratio]{UWTSD-house.pdf}%
               \vfill
               }}}

\begin{document}
\AddToShipoutPicture*{\BackgroundIm}
%----------------------------------------------------------------------------------------
%	POSTER HEADER 
%----------------------------------------------------------------------------------------

% The header is divided into two boxes:
% The first is 75% wide and houses the title, subtitle, names, university/organization and contact information
% The second is 25% wide and houses a logo for your university/organization or a photo of you
% The widths of these boxes can be easily edited to accommodate your content as you see fit



\begin{minipage}[t]{0.60\linewidth}
	\vspace{9.5cm}
	\Huge \color{ku} \textbf{A Risky Coexistence: Examining the Challenges Faced by Autonomous Vehicles and Motorcycles} \color{Black}\\ % Title
	\huge\textit{Research of the Enhancing Self-Driving Car Performance: The
		Potential Dangers of Autonomous Vehicles and Motorcycles}\\[1cm] % Subtitle
	\Large \textbf{Edward S. R. Patch (1801492)}\\[0.5cm] % Author(s)
	\Large Software Engineering and Artificial Intelligence\\[0.4cm] % University/organization

\end{minipage}
%
\begin{minipage}[t]{0.40\linewidth}
	\vspace{9.5cm}
	\flushright
	\color{DarkSlateGray}
	\Large \textbf{Contact Information:}\\
	Waterfront IQ Campus\\
	University of Wales Trinity,\\
	Swansea, SA1 8EW.\\[1cm]
	Email: \texttt{1801492@student.uwtsd.ac.uk} % Email address
\end{minipage}

\vspace{1cm} % A bit of extra whitespace between the header and poster content

%----------------------------------------------------------------------------------------

\begin{multicols}{2} % This is how many columns your poster will be broken into, a portrait poster is generally split into 2 columns

	%----------------------------------------------------------------------------------------
	%	ABSTRACT
	%----------------------------------------------------------------------------------------

	\color{ku} % Navy color for the abstract

	\begin{abstract}

	\end{abstract}

	%----------------------------------------------------------------------------------------
	%	INTRODUCTION
	%----------------------------------------------------------------------------------------

	\color{DarkRed} % SaddleBrown color for the introduction

	\section*{Introduction}
		

	%----------------------------------------------------------------------------------------
	%	OBJECTIVES
	%----------------------------------------------------------------------------------------

	\color{DarkSlateGray} % DarkSlateGray color for the rest of the content

	\section*{Main Objectives}
		\begin{enumerate}
			\item Understanding of what dangers exist with AVs.
			\item Establishing the appropriate datasets to train and test the models.
			\item Pre-processing any datasets to improve the training progress.
			\item Evaluation of the test results to see specific information about where AVs may fail.
		\end{enumerate}

	%----------------------------------------------------------------------------------------
	%	MATERIALS AND METHODS
	%----------------------------------------------------------------------------------------

	\section*{Materials and Methods}
		
	%----------------------------------------------------------------------------------------
	%	RESULTS 
	%----------------------------------------------------------------------------------------

	\section*{Results}


	%----------------------------------------------------------------------------------------
	%	CONCLUSIONS
	%----------------------------------------------------------------------------------------

	\color{DarkRed} % SaddleBrown color for the conclusions to make them stand out

	\section*{Conclusions}


	\color{DarkSlateGray} % Set the color back to DarkSlateGray for the rest of the content

	%----------------------------------------------------------------------------------------
	%	FORTHCOMING RESEARCH
	%----------------------------------------------------------------------------------------

	\section*{Forthcoming Research}

	%----------------------------------------------------------------------------------------
	%	REFERENCES
	%----------------------------------------------------------------------------------------

	\nocite{*} % Print all references regardless of whether they were cited in the poster or not
	\bibliographystyle{plain} % Plain referencing style
	\bibliography{ref} % Use the example bibliography file sample.bib

	%----------------------------------------------------------------------------------------
	%	ACKNOWLEDGEMENTS
	%----------------------------------------------------------------------------------------

	\section*{Acknowledgements}

	%----------------------------------------------------------------------------------------

\end{multicols}
\end{document}
